\documentclass[11pt,a4paper]{report}
\usepackage[utf8]{inputenc}
\usepackage[english]{babel}
\usepackage{mathtools}
\usepackage{amsfonts}
\usepackage{upgreek}
\usepackage{amssymb}
\usepackage{graphicx}
\usepackage[font=footnotesize,labelfont=bf]{caption}
\usepackage[font=footnotesize,labelfont=bf]{subcaption}
\usepackage{lmodern}
\usepackage[left=1.5cm,right=1.5cm,top=1.5cm,bottom=1.5cm]{geometry}
\usepackage{fancyhdr}
\usepackage{eurosym}
\usepackage{dcolumn}% Align table columns on decimal point
\usepackage{bm}% bold math
%\usepackage{citesort}
\usepackage{booktabs}
\usepackage{multirow}
\usepackage{framed}
\usepackage{ulem}
\usepackage[framed,numbered,autolinebreaks,useliterate]{mcode}
\usepackage{wrapfig}
\usepackage{url}
\usepackage{etoolbox}
% \makeatletter
% \patchcmd{\chapter}{\if@openright\cleardoublepage\else\clearpage\fi}{}{}{}
% \makeatother

%Stuff I have added

%Reformat some spacing and sizing around titles
\usepackage{titlesec}
\titleformat{\chapter}[display]
{\normalfont\Large\bfseries}{\chaptertitlename\ \thechapter}{20pt}{\LARGE}
\titlespacing{\chapter}{0pt}{0pt}{12pt}

%New command to get rid of the "Chapter X" at the beginning of every chapter while maintaining
%a chapter count for the table of contents
%1st input is the counter for the chapter, second is the chapter name
\newcommand{\mychapter}[2]
{
    \setcounter{chapter}{#1}
    \setcounter{section}{0}
    \chapter*{#2}
    \addcontentsline{toc}{chapter}{#2}
}

\usepackage{float}

\usepackage{enumitem}

%Get rid of auto indent for paragraphs
\newlength\tindent
\setlength{\tindent}{\parindent}
\setlength{\parindent}{0pt}
\renewcommand{\indent}{\hspace*{\tindent}}

%End of stuff I have added

\renewcommand{\arraycolsep}{2pt}
\newcommand{\eq}{Equation~}
\newcommand{\eqs}{Equations~}
\newcommand{\fig}{Figure~}
\newcommand{\figs}{Figures~}
\newcommand{\tab}{Table~}
\newcommand{\tabs}{Tables~}
\newcommand{\kwm}{k-\omega^2m}
\setcounter{tocdepth}{2}
\setcounter{secnumdepth}{3}

\graphicspath{{pictures/}}

\begin{document}
\begin{titlepage}
\vspace*{\fill}

\begin{center}

\Huge{\textbf{EE4-45:}\\ Wavelets And Applications}\\
\vspace{1cm}
\Huge{Coursework}

\end{center}

\centering{
\begin{tabular}{rl}
\\
{\bf Name:} & {Meng Kiang SEAH}
\\
{\bf CID:} & {00699092}
\\
{\bf Date:} & {27\textsuperscript{th} March 2017}

\end{tabular}}
\vspace*{\fill}

\end{titlepage}

\pagenumbering{roman}
\tableofcontents
\newpage

\mychapter{1}{Exercise 1}

\pagenumbering{arabic}
\setcounter{page}{1}

From the question sheet itself \cite{question}, it states that to reproduce polynomials of degree $N$, a scaling function producing wavelets with $N+1$ vanishing moments is needed. Thus, to reproduce polynomials of maximum degree 3, then the Daubechie scaling function of order 4 is needed (\mcode{'db4'}). The $\varphi({t})$ output from the \mcode{wavefun} function is graphed in Figure \ref{fig:e1_1}.

\begin{figure}[!ht]
    \centering
    \includegraphics[width=\textwidth]{../pictures/ex1_1.eps}
    \caption{Graph of $\varphi({t})$ from a sampling .}
    \label{fig:e1_1}
\end{figure}

To compute the coefficients, the Equation \ref{eq:coefficient} is used \cite{question}. However, the function \mcode{wavefun} only has returned $\varphi(t)$. The trick is knowing that Daubechie filters are orthogonal, meaning that for their case, $\varphi(t)$ can be used as $\tilde{\varphi}(t)$. Taking the dot product gives the coefficient. However, from Equation \ref{eq:constant} from Page 68 of the notes \cite{notes}, the coefficients must be normalised, by dividing by $T$.

\begin{align}
    c_{m,n} &= \langle t^m, \tilde{\varphi}(t - n)\rangle\label{eq:coefficient}\\
    c_{m,n} &= \frac{1}{T} \int_{-\infty}^{\infty} t^m \tilde{\varphi} \left( \frac{t}{T} -n \right) \,dt \label{eq:constant}
\end{align}

Thus, each shifted kernel is multiplied by the signal and the result divided by 64, and stored. To reconstruct the signal, the coefficient is multiplied by the shifted kernel and summed. This was done for signals ranging from order $m \in [0; 4]$. The results are shown in Figure \ref{fig:ex1_2}. This shows the original signal, the reconstructed signal, and the shifted kernels that were summed to recreate the signal.
\\\\
At first glance, it would seem that the reconstruction works even up to $m=4$ specifically when looking in the middle of the signal, such as $t \in [7; 22]$. To investigate this further, the error can be calculated by summing the absolute value of the difference between the reconstructed signals in the region where they seem to be aligned. The results are in Table \ref{tbl:ex1}, where the error can be seen to be much greater for $m=4$.

\begin{figure}[!ht]
    \captionsetup[subfigure]{position=b}
    \centering
    \begin{subfigure}{0.49\textwidth}
        \includegraphics[width=\textwidth]{../pictures/ex1_2_0.eps}
        \caption{Signal of Order 0.}
        \label{fig:ex1_2_0}
    \end{subfigure}
    ~
    \begin{subfigure}{0.49\textwidth}
        \includegraphics[width=\textwidth]{../pictures/ex1_2_1.eps}
        \caption{Signal of Order 1.}
        \label{fig:ex1_2_1}
    \end{subfigure}
    \\
    \begin{subfigure}{0.49\textwidth}
        \includegraphics[width=\textwidth]{../pictures/ex1_2_2.eps}
        \caption{Signal of Order 2.}
        \label{fig:ex1_2_2}
    \end{subfigure}
    ~
    \begin{subfigure}{0.49\textwidth}
        \includegraphics[width=\textwidth]{../pictures/ex1_2_3.eps}
        \caption{Signal of Order 3.}
        \label{fig:ex1_2_3}
    \end{subfigure}
    \\
    \begin{subfigure}{0.75\textwidth}
        \includegraphics[width=\textwidth]{../pictures/ex1_2_4.eps}
        \caption{Signal of Order 4.}
        \label{fig:ex1_2_4}
    \end{subfigure}

    \caption{Reconstruction of signals of various orders with a Daubechie sampling kernel of order 4.}
    \label{fig:ex1_2}
\end{figure}

\begin{table}[!ht]
    \centering
    \begin{tabular}{|c|r|r|r|r|r|}
        \hline
        $m$     & \textbf{0} & \textbf{1} & \textbf{2} & \textbf{3} & \textbf{4}\\ \hline
        Error   & 0.2184     & 0.0048     & 0.0050     & 0.0255     & 233.5417\\ \hline
    \end{tabular}
    \caption{Absolute value of the reconstruction error.}
    \label{tbl:ex1}
\end{table}


\mychapter{2}{Exercise 2}
Left for later.

\mychapter{3}{Exercise 3}
Following exactly the equations set up in the question sheet \cite{question}, the function \mcode{annihilatingFilter} was written. This was then applied to the \mcode{tau} variable obtained from the \texttt{tau.mat} file given to us. The results are as found in Table \ref{tbl:ex3_1}. The $h[n]$ values are found in Table \ref{tbl:ex3_2}.

\begin{table}[!ht]
    \centering
    \begin{tabular}{|c|c|c|}
        \hline
        $k$     & $t_k$     & $a_k$ \\ \hline
        0       & 15.3750   & 0.7800\\ \hline
        1       & 14.2500   & 1.3200\\ \hline
    \end{tabular}
    \caption{Results of the annihilating filter applied to \texttt{tau.mat}.}
    \label{tbl:ex3_1}
\end{table}

\begin{table}[!ht]
    \centering
    \begin{tabular}{|c|c|c|}
        \hline
        $h[0]$     & $h[1]$     & $h[2]$ \\ \hline
        1.0000       & -29.6250  & 219.0937\\ \hline
    \end{tabular}
    \caption{The annihilating filter coefficients.}
    \label{tbl:ex3_2}
\end{table}

\mychapter{4}{Exercise 4}

First, the Dirac stream was created. The value of $K=2$ means that there are two Diracs in the stream. Their location and amplitude are found in Table \ref{tbl:ex4_1}.

\begin{table}[!ht]
    \centering
    \begin{tabular}{|c|c|}
        \hline
        \textbf{Sample Number} & \textbf{Amplitude} \\ \hline
        517                    & 6.98 \\ \hline
        1569                   & 2.67 \\ \hline
    \end{tabular}
    \caption{The location of the Diracs and their amplitude.}
    \label{tbl:ex4_1}
\end{table}

Then, the signal $x(t)$ was sampled using the same 4th Order Daubechie function as Exercise 1. This is because the notes \cite{notes} on Page 70 state that the $\varphi(t)$ used must be able to reproduce polynomials of maximum degree $N \leq 2K -1$. For $K=2$, this means $N \leq 3$, exactly as Exercise 1. This produced the 26-coefficient vector (as the support of the kernel $L = 26$).
\\\\
The coefficients from Exercise 1 were 4 sets of 26 coefficients, corresponding to the orders $m \in [0;3]$. The dot product between each order's coefficients and the sampled coefficients was computed. The resulting values (4) created the $s[m]$ values.
\\\\
This was fed into the annihilating filter. The $t_k$ resulting values and $a_k$ values should correspond to the Diracs in the original signal, as shown in Table \ref{tbl:ex4_2}. However, while the amplitude values line up, the time ones do not. This is because they correspond to the value in seconds, and the input values are the sample number, given that 64 samples happen a second. Also, the index in MATLAB versus the equation are different. Multiplying the $t_k$ by 64, and adding one as in Table \ref{tbl:ex4_3} gives the correct values. The original signal and the reconstructed signals are shown in Figure \ref{fig:ex4}.

\begin{table}[!ht]
    \parbox{.45\linewidth}{
    \centering
    \begin{tabular}{|c|c|c|}
        \hline
        $k$     & $t_k$/seconds     & $a_k$ \\ \hline
        0       & 24.5000   & 2.6700\\ \hline
        1       & 8.0625   & 6.9800\\ \hline
    \end{tabular}
    \caption{Results of the annihilating filter: location and amplitude of Diracs.}
    \label{tbl:ex4_2}
    }
    \hfill
    \parbox{.45\linewidth}{
    \centering
    \begin{tabular}{|c|c|}
        \hline
        \textbf{Sample Number} \\ \hline
        517   \\ \hline
        1569  \\ \hline
    \end{tabular}
    \caption{The sample number location of the reconstructed Diracs.}
    \label{tbl:ex4_3}
    }
\end{table}

\begin{figure}[!ht]
    \centering
    \includegraphics[width=.8\textwidth]{../pictures/ex4.eps}
    \caption{The original and reconstructed Dirac streams.}
    \label{fig:ex4}
\end{figure}

\mychapter{5}{Exercise 5}
Applying the same method as used in Exercise 4, the sampled signal can be analysed to determine the location of the Diracs and their amplitude. The results are found in Table \ref{tbl:ex5} and the signal plotted is in Figure \ref{fig:ex5}.

\begin{table}[!ht]
    \centering
    \begin{tabular}{|c|c|c|c|}
        \hline
        \textbf{Dirac Number} & \textbf{Time (s)} & \textbf{Sample Number} & \textbf{Amplitude}\\ \hline
        1                     & 14.3750           & 921                    & 2.6300\\ \hline
        2                     & 17.5000           & 1121                   & 1.4800\\ \hline
    \end{tabular}
    \caption{The location of the Diracs and their amplitude from the unknown signal.}
    \label{tbl:ex5}
\end{table}

\begin{figure}[!ht]
    \centering
    \includegraphics[width=\textwidth]{../pictures/ex5.eps}
    \caption{The reconstructed Dirac stream from the sampled data.}
    \label{fig:ex5}
\end{figure}



\mychapter{6}{Exercise 6}
Something.

\mychapter{7}{Exercise 7}
Something.

\mychapter{8}{Exercise 8}
Something.


% \[ f_{X}(x) = \left\{
% 	\begin{array}{l l}
% 		1 & \quad 0 \leq t \leq 1\\
% 		0 & \quad \text{otherwise}\\
% 	\end{array}
% \right.\]
%
% \resizebox{\textwidth}{!}{%
%   \begin{tabular}{ lll }
%     \parbox{0.3\textwidth}{\begin{align}
%       	m &= \mathbb{E}[X]\notag\\
%       	&= \int_{-\infty}^{+\infty} xf_{X}(x) \,dx \notag\\
%       	&= \int_{0}^{1}x \,dx \notag\\
%       	&= \left[\frac{x^2}{2}\right]_{0}^{1}\notag\\
%       	&= \frac{1}{2}\label{eq:rvunimu}
%     \end{align}}
%     &
%     \parbox{0.3\textwidth}{\begin{align*}
%     \sigma &= \sqrt{E[X^2] - (E[X])^2}\\
%   	E[X^2] &= \int_{-\infty}^{+\infty} x^2 f_{X}(x) \,dx\\
%   	&= \int_{0}^{1}x^2 \,dx\\
%     &= \left[\frac{x^3}{3}\right]_{0}^{1}\\
%   	&= \frac{1}{3}
%     \end{align*}}
%     &
%     \parbox{0.4\textwidth}{\begin{align}
%     (E[X])^2 &= \left(\frac{1}{2}\right)^2\notag\\
%   	&=\frac{1}{4}\notag\\
%   	\sqrt{E[X^2] - (E[X])^2} &= \sqrt{\frac{1}{3} - \frac{1}{4}}\notag\\
%   	&=\sqrt{\frac{1}{12}}\notag\\
%   	&\approx 0.2887 \label{eq:rvunisigma}
%     \end{align}}
%   \end{tabular}
%   }

% \begin{wrapfigure}{r}{0.4\textwidth}
%       \begin{tabular}{|l|l|}
%         \hline
%         \textbf{Sample Mean ($\hat{m}$)} & 0.4982\\ \hline
%         \textbf{Sample Standard Deviation ($\hat{\sigma}$)} & 0.2862 \\ \hline
%     \end{tabular}
% \end{wrapfigure}


\newpage
\mychapter{9}{References}
\begingroup
   \def\chapter*#1{}
\bibliographystyle{biblio}
\bibliography{wavelets}
\endgroup

\mychapter{10}{Appendix}
\renewcommand\thesection{\Alph{section}}
\section{Exercise 1}
\lstinputlisting{../ex1.m}
\newpage

% \section{Exercise 2}
% \lstinputlisting{../ex2.m}
% \newpage

\section{Exercise 3}
\subsection{Annihilating Filter Function}
\lstinputlisting{../annihilatingFilter.m}
\newpage

\subsection{Actual Exercise Script}
\lstinputlisting{../ex3.m}
\newpage

\section{Exercise 4}
\lstinputlisting{../ex4.m}
\newpage

\section{Exercise 5}
\lstinputlisting{../ex5.m}
\newpage

\end{document}
