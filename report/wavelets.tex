\documentclass[11pt,a4paper]{report}
\usepackage[utf8]{inputenc}
\usepackage[english]{babel}
\usepackage{mathtools}
\usepackage{amsfonts}
\usepackage{upgreek}
\usepackage{amssymb}
\usepackage{graphicx}
\usepackage[font=footnotesize,labelfont=bf]{caption}
\usepackage[font=footnotesize,labelfont=bf]{subcaption}
\usepackage{lmodern}
\usepackage[left=1.5cm,right=1.5cm,top=1.5cm,bottom=1.5cm]{geometry}
\usepackage{fancyhdr}
\usepackage{eurosym}
\usepackage{dcolumn}% Align table columns on decimal point
\usepackage{bm}% bold math
\usepackage{booktabs}
\usepackage{multirow}
\usepackage{framed}
\usepackage{ulem}
\usepackage[framed,numbered,autolinebreaks,useliterate]{mcode}
\usepackage{wrapfig}
\usepackage{url}
\usepackage{etoolbox}

%Stuff I have added

%Reformat some spacing and sizing around titles
\usepackage{titlesec}
\titleformat{\chapter}[display]
{\normalfont\Large\bfseries}{\chaptertitlename\ \thechapter}{20pt}{\LARGE}
\titlespacing{\chapter}{0pt}{0pt}{12pt}

%New command to get rid of the "Chapter X" at the beginning of every chapter while maintaining
%a chapter count for the table of contents
%1st input is the counter for the chapter, second is the chapter name
\newcommand{\mychapter}[2]
{
    \setcounter{chapter}{#1}
    \setcounter{section}{0}
    \chapter*{#2}
    \addcontentsline{toc}{chapter}{#2}
}

\usepackage{float}

\usepackage{enumitem}

%Get rid of auto indent for paragraphs
\newlength\tindent
\setlength{\tindent}{\parindent}
\setlength{\parindent}{0pt}
\renewcommand{\indent}{\hspace*{\tindent}}

%End of stuff I have added

\renewcommand{\arraycolsep}{2pt}
\newcommand{\eq}{Equation~}
\newcommand{\eqs}{Equations~}
\newcommand{\fig}{Figure~}
\newcommand{\figs}{Figures~}
\newcommand{\tab}{Table~}
\newcommand{\tabs}{Tables~}
\newcommand{\kwm}{k-\omega^2m}
\setcounter{tocdepth}{2}
\setcounter{secnumdepth}{3}

\graphicspath{{pictures/}}

\begin{document}
\begin{titlepage}
\vspace*{\fill}

\begin{center}

\Huge{\textbf{EE4-45:}\\ Wavelets And Applications}\\
\vspace{1cm}
\Huge{Coursework}

\end{center}

\centering{
\begin{tabular}{rl}
\\
{\bf Name:} & {Meng Kiang SEAH}
\\
{\bf CID:} & {00699092}
\\
{\bf Date:} & {27\textsuperscript{th} March 2017}

\end{tabular}}
\vspace*{\fill}

\end{titlepage}

\pagenumbering{roman}
\tableofcontents
\newpage

\mychapter{1}{Exercise 1}

\pagenumbering{arabic}
\setcounter{page}{1}

From the question sheet itself \cite{question}, it states that to reproduce polynomials of degree $N$, a scaling function producing wavelets with $N+1$ vanishing moments is needed. Thus, to reproduce polynomials of maximum degree 3, then the Daubechies scaling function of order 4 is needed (\mcode{'db4'}). The $\varphi({t})$ output from the \mcode{wavefun} function is graphed in Figure \ref{fig:e1_1}.

\begin{figure}[!ht]
    \centering
    \includegraphics[width=\textwidth]{../pictures/ex1_1.eps}
    \caption{Graph of $\varphi({t})$ from a sampling .}
    \label{fig:e1_1}
\end{figure}

To compute the coefficients, the Equation \ref{eq:coefficient} is used \cite{question}. However, the function \mcode{wavefun} only has returned $\varphi(t)$. The trick is knowing that Daubechies filters are orthogonal, meaning that for their case, $\varphi(t)$ can be used as $\tilde{\varphi}(t)$. Taking the dot product gives the coefficient. However, from Equation \ref{eq:constant} from Page 68 of the notes \cite{notes}, the coefficients must be normalised, by dividing by $T$.

\begin{align}
    c_{m,n} &= \langle t^m, \tilde{\varphi}(t - n)\rangle\label{eq:coefficient}\\
    c_{m,n} &= \frac{1}{T} \int_{-\infty}^{\infty} t^m \tilde{\varphi} \left( \frac{t}{T} -n \right) \,dt \label{eq:constant}
\end{align}

Thus, each shifted kernel is multiplied by the signal and the result divided by 64, and stored. To reconstruct the signal, the coefficient is multiplied by the shifted kernel and summed. This was done for signals ranging from order $m \in [0; 4]$. The results are shown in Figure \ref{fig:ex1_2}. This shows the original signal, the reconstructed signal, and the shifted kernels that were summed to recreate the signal.
\\\\
At first glance, it would seem that the reconstruction works even up to $m=4$ specifically when looking in the middle of the signal, such as $t \in [7; 22]$. To investigate this further, the error can be calculated by summing the absolute value of the difference between the reconstructed signals in the region where they seem to be aligned. The results are in Table \ref{tbl:ex1}, where the error can be seen to be much greater for $m=4$.

\begin{figure}[!ht]
    \captionsetup[subfigure]{position=b}
    \centering
    \begin{subfigure}{0.49\textwidth}
        \includegraphics[width=\textwidth]{../pictures/ex1_2_0.eps}
        \caption{Signal of Order 0.}
        \label{fig:ex1_2_0}
    \end{subfigure}
    ~
    \begin{subfigure}{0.49\textwidth}
        \includegraphics[width=\textwidth]{../pictures/ex1_2_1.eps}
        \caption{Signal of Order 1.}
        \label{fig:ex1_2_1}
    \end{subfigure}
    \\
    \begin{subfigure}{0.49\textwidth}
        \includegraphics[width=\textwidth]{../pictures/ex1_2_2.eps}
        \caption{Signal of Order 2.}
        \label{fig:ex1_2_2}
    \end{subfigure}
    ~
    \begin{subfigure}{0.49\textwidth}
        \includegraphics[width=\textwidth]{../pictures/ex1_2_3.eps}
        \caption{Signal of Order 3.}
        \label{fig:ex1_2_3}
    \end{subfigure}
    \\
    \begin{subfigure}{0.75\textwidth}
        \includegraphics[width=\textwidth]{../pictures/ex1_2_4.eps}
        \caption{Signal of Order 4.}
        \label{fig:ex1_2_4}
    \end{subfigure}

    \caption{Reconstruction of signals of various orders with a Daubechies sampling kernel of order 4.}
    \label{fig:ex1_2}
\end{figure}

\begin{table}[!ht]
    \centering
    \begin{tabular}{|c|r|r|r|r|r|}
        \hline
        $m$     & \textbf{0} & \textbf{1} & \textbf{2} & \textbf{3} & \textbf{4}\\ \hline
        Error   & 0.2184     & 0.0048     & 0.0050     & 0.0255     & 233.5417\\ \hline
    \end{tabular}
    \caption{Absolute value of the reconstruction error.}
    \label{tbl:ex1}
\end{table}

\mychapter{3}{Exercise 3}
Following exactly the equations set up in the question sheet \cite{question}, the function \mcode{annihilatingFilter} was written. This was then applied to the \mcode{tau} variable obtained from the \texttt{tau.mat} file given to us. The results are as found in Table \ref{tbl:ex3_1}. The $h[n]$ values are found in Table \ref{tbl:ex3_2}.

\begin{table}[!ht]
    \centering
    \begin{tabular}{|c|c|c|}
        \hline
        $k$     & $t_k$     & $a_k$ \\ \hline
        0       & 15.3750   & 0.7800\\ \hline
        1       & 14.2500   & 1.3200\\ \hline
    \end{tabular}
    \caption{Results of the annihilating filter applied to \texttt{tau.mat}.}
    \label{tbl:ex3_1}
\end{table}

\begin{table}[!ht]
    \centering
    \begin{tabular}{|c|c|c|}
        \hline
        $h[0]$     & $h[1]$     & $h[2]$ \\ \hline
        1.0000       & -29.6250  & 219.0937\\ \hline
    \end{tabular}
    \caption{The annihilating filter coefficients.}
    \label{tbl:ex3_2}
\end{table}

\mychapter{4}{Exercise 4}

First, the Dirac stream was created. The value of $K=2$ means that there are two Diracs in the stream. Their location and amplitude are found in Table \ref{tbl:ex4_1}.

\begin{table}[!ht]
    \centering
    \begin{tabular}{|c|c|}
        \hline
        \textbf{Sample Number} & \textbf{Amplitude} \\ \hline
        517                    & 6.98 \\ \hline
        1569                   & 2.67 \\ \hline
    \end{tabular}
    \caption{The location of the Diracs and their amplitude.}
    \label{tbl:ex4_1}
\end{table}

Then, the signal $x(t)$ was sampled using the same 4th Order Daubechies function as Exercise 1. This is because the notes \cite{notes} on Page 70 state that the $\varphi(t)$ used must be able to reproduce polynomials of maximum degree $N \geq 2K -1$. For $K=2$, this means $N \geq 3$, exactly as Exercise 1. This produced the 26-coefficient vector (as the support of the kernel $L = 7$).
\\\\
The coefficients from Exercise 1 were 4 sets of 26 coefficients, corresponding to the orders $m \in [0;3]$. The dot product between each order's coefficients and the sampled coefficients was computed. The resulting values (4) created the $s[m]$ values.
\\\\
This was fed into the annihilating filter. The $t_k$ resulting values and $a_k$ values should correspond to the Diracs in the original signal, as shown in Table \ref{tbl:ex4_2}. However, while the amplitude values line up, the time ones do not. This is because they correspond to the value in seconds, and the input values are the sample number, given that 64 samples happen a second. Also, the index in MATLAB versus the equation are different. Multiplying the $t_k$ by 64, and adding one as in Table \ref{tbl:ex4_3} gives the correct values. The original signal and the reconstructed signals are shown in Figure \ref{fig:ex4}.

\begin{table}[!ht]
    \parbox{.45\linewidth}{
    \centering
    \begin{tabular}{|c|c|c|}
        \hline
        $k$     & $t_k$/seconds     & $a_k$ \\ \hline
        0       & 24.5000   & 2.6700\\ \hline
        1       & 8.0625   & 6.9800\\ \hline
    \end{tabular}
    \caption{Results of the annihilating filter: location and amplitude of Diracs.}
    \label{tbl:ex4_2}
    }
    \hfill
    \parbox{.45\linewidth}{
    \centering
    \begin{tabular}{|c|c|}
        \hline
        \textbf{Sample Number} \\ \hline
        517   \\ \hline
        1569  \\ \hline
    \end{tabular}
    \caption{The sample number location of the reconstructed Diracs.}
    \label{tbl:ex4_3}
    }
\end{table}

\begin{figure}[!ht]
    \centering
    \includegraphics[width=.8\textwidth]{../pictures/ex4.eps}
    \caption{The original and reconstructed Dirac streams.}
    \label{fig:ex4}
\end{figure}

\mychapter{5}{Exercise 5}
Applying the same method as used in Exercise 4, the sampled signal can be analysed to determine the location of the Diracs and their amplitude. The results are found in Table \ref{tbl:ex5} and the signal plotted is in Figure \ref{fig:ex5}. One thing to note is that the signal from \texttt{samples.mat} were 32 elements long. However, only 26 coefficients were used, as $n \in [0; 32 - L]$ meant that since $L=7$, $n \in [0; 25]$. However, this made no difference, as the coefficients at the end that were left out were all $0$ anyway.

\begin{table}[!ht]
    \centering
    \begin{tabular}{|c|c|c|c|}
        \hline
        \textbf{Dirac Number} & \textbf{Time (s)} & \textbf{Sample Number} & \textbf{Amplitude}\\ \hline
        1                     & 14.3750           & 921                    & 2.6300\\ \hline
        2                     & 17.5000           & 1121                   & 1.4800\\ \hline
    \end{tabular}
    \caption{The location of the Diracs and their amplitude from the unknown signal.}
    \label{tbl:ex5}
\end{table}

\begin{figure}[!ht]
    \centering
    \includegraphics[width=\textwidth]{../pictures/ex5.eps}
    \caption{The reconstructed Dirac stream from the sampled data.}
    \label{fig:ex5}
\end{figure}

\mychapter{6}{Exercise 6}
The function \mcode{daubechieMoments} was written to take in a signal, and an $N$ value, before returning the moments of $s[m]$ where $m\in[0;N-1]$. It does this by first generating a Daubechies function, creating the various $t^m$ values, and obtaining the coefficients through the sampling kernel with shifts. Then, it samples the signal using that same sampling kernel with various shifts. Those samples and the coefficients are combined to form the various moments. This follows exactly the same procedure as in previous exercises.
\\\\
The function was created to streamline the generation process of the data with varying $N$ and noise variance. Knowing that $N > 2K$ from the question sheet \cite{question}, and knowing that $K =2$, this means that $N > 4$. Hence, $N \in [5;8]$ was chosen. After the moment was recorded, noise was added to it. 5 different levels of noise were added, and each result was saved. The variances chosen were $\sigma^2 \in \{0.001, 0.01, 0.1, 1, 10\}$.
\\\\
While reproducing all the coefficients and signals and various outputs are not so interesting, Figure \ref{fig:ex6} shows one picture of the kernel, and the various moments with noise. This is for $N = 5$. To save space, the rest can be found in the Appendix.

\begin{figure}[!ht]
    \centering
    \includegraphics[width=\textwidth]{../pictures/ex6_moments5.eps}
    \caption{Noisy moments and kernel for $N = 5$.}
    \label{fig:ex6}
\end{figure}

\mychapter{7}{Exercise 7}
Using the moments from $N = 5$, the various moments with noise can be processed through the annihilating filter. But there can be improvements to the annihilating filter. The total least-squares (TLS) approach involves computing $h[n]$ differently by using Singular Value Decomposition, or SVD. A larger Toeplitz matrix, $S$ is broken down into the form $U\Lambda V^T$, where the last column of $V$ is $h[n]$. Then, $t_k$ and $a_k$ are found in the same way.
\\\\
In addition to that, the $S$ matrix can be improved before TLS through the Cadzow method. This involves taking the SVD again, but this time, $U\Lambda V^T$ is modified by keeping only the $K$ largest eigenvalues in $\Lambda$. This becomes $\Lambda'$, from which $S$ is reconstituted. This is repeated as many times as needed, before TLS, and the normal methods are used.
\\\\
There were 4 different $N$ values explored. In addition, the moment for each one was exposed to 5 different noise variances. Finally, the three methods of filtering were used: normal annihilation filter, TLS, and TLS with Cadzow. All of these are shown in the Figures \ref{fig:ex7_5} to \ref{fig:ex7_tlscad_8}. Only Figure \ref{fig:ex7_5} is included in the main body, to avoid clutter. The rest can be found in the Appendix.
\\\\
There are a number of observations here. For the normal annihilation filter, the result was always two Diracs detected. However, with added noise, the time position started to stray from the correct value. Additionally, there was also a noticeable decrease in the amplitude detected. This is expected, as the error from the noise must manifest itself in some way. What is interesting is that increasing $N$ did not seem to improve the performance directly, and appeared to cause attenuation to the later Dirac.
\\\\
As for the TLS results, it seemed more resistant to noise, until $\sigma^2 = 10$, where an entire Dirac was lost, and the single one was neither in the right place, nor at the right amplitude. Increasing $N$ meant greater attenuation, although that did allow the second Dirac to be detected at higher noise levels, despite the wrong location and amplitude. Adding the Cadzow improvement did not seem to do much, but it was only iterated through twice. Due to time and space constraints, this effect was not investigated, but could offer an avenue for further exploration.

\begin{figure}[!ht]
    \captionsetup[subfigure]{position=b}
    \centering
    \begin{subfigure}{0.49\textwidth}
        \includegraphics[width=\textwidth]{../pictures/originalDirac.eps}
        \caption{Actual Diracs.}
        \label{fig:ex7_5_0}
    \end{subfigure}
    ~
    \begin{subfigure}{0.49\textwidth}
        \includegraphics[width=\textwidth]{../pictures/ex7_5_1.eps}
        \caption{$\sigma^2 = 0.001$.}
        \label{fig:ex7_5_1}
    \end{subfigure}
    \\
    \begin{subfigure}{0.49\textwidth}
        \includegraphics[width=\textwidth]{../pictures/ex7_5_2.eps}
        \caption{$\sigma^2 = 0.01$.}
        \label{fig:ex7_5_2}
    \end{subfigure}
    ~
    \begin{subfigure}{0.49\textwidth}
        \includegraphics[width=\textwidth]{../pictures/ex7_5_3.eps}
        \caption{$\sigma^2 = 0.1$.}
        \label{fig:ex7_5_3}
    \end{subfigure}
    \\
    \begin{subfigure}{0.49\textwidth}
        \includegraphics[width=\textwidth]{../pictures/ex7_5_4.eps}
        \caption{$\sigma^2 = 1$.}
        \label{fig:ex7_5_4}
    \end{subfigure}
    ~
    \begin{subfigure}{0.49\textwidth}
        \includegraphics[width=\textwidth]{../pictures/ex7_5_5.eps}
        \caption{$\sigma^2 = 10$.}
        \label{fig:ex7_5_5}
    \end{subfigure}

    \caption{Estimate of Diracs with sampling when $N = 5$, with normal annihilation filter.}
    \label{fig:ex7_5}
\end{figure}

\mychapter{8}{Exercise 8}
There are a number of things to take note here which help understand the information. First, $f_i(x,y)$ is the value of the pixel in the $x$ and $y$ location of the view from the $i$-th camera. However, the lens of the camera means this is put through some sampling kernel, which results in Equation \ref{eq:S}. This need not be calculated because it is exactly the output from the camera, meaning it is the actual values to work with.

\begin{align}
    S^{(i)}_{m,n} &= \langle f_i (x, y), \varphi(x -m, y - n) \rangle \label{eq:S}\\
    m_{p,q} &= \sum_{m} \sum_{n} c^{p, q}_{m, n} S_{m,n} \label{eq:m}
\end{align}

To calculate the barycenters, $(\bar{x},\bar{y})$, only $m_{1,0}$, $m_{0,1}$, and $m_{0,0}$ are needed. From Equation \ref{eq:m}, $m_{m,n}$ is simply the element-wise multiplication of all the coefficients with all of the camera outputs, summed across both dimensions. Thus, the barycenters are calculated for each picture, and the resultant $(dx_i, dy_i)$ are easily found to input into the fusion function.
\\\\
An example of an output is shown in Figure \ref{fig:ex8_30}, showing a low-resolution and a super-resolved picture side by side. The difference is clearly visible, and the PSNR is given as 24.26. However, when taking the $S^{(i)}_{m,n}$ values, noise was removed as anything below a certain limit was set to 0, and this affects the output PSNR. Figure \ref{fig:ex8_2} shows the relationship between the noise threshold and the resultant PSNR. The peak value is at a threshold of 0.29 and a maximum PSNR of 24.4677. Figure \ref{fig:ex8_29} shows this.

\begin{figure}[!ht]
    \centering
    \includegraphics[width=\textwidth]{../pictures/ex8_30.eps}
    \caption{Example super-resolved picture with threshold at 0.30.}
    \label{fig:ex8_30}
\end{figure}

\begin{figure}[!ht]
    \centering
    \includegraphics[width=\textwidth]{../pictures/ex8_2.eps}
    \caption{Plot of PSNR vs. Noise Threshold.}
    \label{fig:ex8_2}
\end{figure}

\begin{figure}[!ht]
    \centering
    \includegraphics[width=\textwidth]{../pictures/ex8_29.eps}
    \caption{Output with Noise Threshold at 0.29, maximum PSNR.}
    \label{fig:ex8_29}
\end{figure}

\newpage
\mychapter{9}{References}
\begingroup
   \def\chapter*#1{}
\bibliographystyle{biblio}
\bibliography{wavelets}
\endgroup

\mychapter{10}{Appendix}
\renewcommand\thesection{\Alph{section}}
\section{Exercise 1 Code}
\lstinputlisting{../ex1.m}
\newpage

\section{Exercise 3 Code}
\subsection{Annihilating Filter Function}
\lstinputlisting{../annihilatingFilter.m}
\newpage

\subsection{Actual Exercise Script}
\lstinputlisting{../ex3.m}
\newpage

\section{Exercise 4 Code}
\lstinputlisting{../ex4.m}
\newpage

\section{Exercise 5 Code}
\lstinputlisting{../ex5.m}
\newpage

\section{Exercise 6 Code}
\subsection{Moment Generating Function}
\lstinputlisting{../daubechieMoments.m}
\newpage

\subsection{Actual Exercise Script}
\lstinputlisting{../ex6.m}
\newpage

\subsection{Pictures!}

\begin{figure}[!ht]
    \centering
    \includegraphics[width=\textwidth]{../pictures/ex6_moments6.eps}
    \caption{Noisy moments and kernel for $N = 6$.}
\end{figure}

\begin{figure}[!ht]
    \centering
    \includegraphics[width=\textwidth]{../pictures/ex6_moments7.eps}
    \caption{Noisy moments and kernel for $N = 7$.}
\end{figure}

\begin{figure}[!ht]
    \centering
    \includegraphics[width=\textwidth]{../pictures/ex6_moments8.eps}
    \caption{Noisy moments and kernel for $N = 8$.}
\end{figure}

\section{Exercise 7 Code and Plots}
\subsection{Original Dirac Function}
\lstinputlisting{../originalDirac.m}
\newpage

\subsection{Annihilation Function with TLS}
\lstinputlisting{../annihilatingFilterTLS.m}
\newpage

\subsection{Annihilation Function with TLS and Cadzow}
\lstinputlisting{../annihilatingFilterTLSCadzow.m}
\newpage

\subsection{Script using Normal Annihilation Filter}
\lstinputlisting{../ex7_none.m}
\newpage

\subsection{Script using Annihilation Filter with TLS}
\lstinputlisting{../ex7_tls.m}
\newpage

\subsection{Script using Annihilation Filter with TLS and Cadzow}
\lstinputlisting{../ex7_tlscad.m}
\newpage

\subsection{Plots of Predicted Dirac Locations}
\begin{figure}[!ht]
    \captionsetup[subfigure]{position=b}
    \centering
    \begin{subfigure}{0.49\textwidth}
        \includegraphics[width=\textwidth]{../pictures/originalDirac.eps}
        \caption{Actual Diracs.}
        \label{fig:ex7_6_0}
    \end{subfigure}
    ~
    \begin{subfigure}{0.49\textwidth}
        \includegraphics[width=\textwidth]{../pictures/ex7_6_1.eps}
        \caption{$\sigma^2 = 0.001$.}
        \label{fig:ex7_6_1}
    \end{subfigure}
    \\
    \begin{subfigure}{0.49\textwidth}
        \includegraphics[width=\textwidth]{../pictures/ex7_6_2.eps}
        \caption{$\sigma^2 = 0.01$.}
        \label{fig:ex7_6_2}
    \end{subfigure}
    ~
    \begin{subfigure}{0.49\textwidth}
        \includegraphics[width=\textwidth]{../pictures/ex7_6_3.eps}
        \caption{$\sigma^2 = 0.1$.}
        \label{fig:ex7_6_3}
    \end{subfigure}
    \\
    \begin{subfigure}{0.49\textwidth}
        \includegraphics[width=\textwidth]{../pictures/ex7_6_4.eps}
        \caption{$\sigma^2 = 1$.}
        \label{fig:ex7_6_4}
    \end{subfigure}
    ~
    \begin{subfigure}{0.49\textwidth}
        \includegraphics[width=\textwidth]{../pictures/ex7_6_5.eps}
        \caption{$\sigma^2 = 10$.}
        \label{fig:ex7_6_5}
    \end{subfigure}

    \caption{Estimate of Diracs with sampling when $N = 6$, with normal annihilation filter.}
    \label{fig:ex7_6}
\end{figure}

\begin{figure}[H]
    \captionsetup[subfigure]{position=b}
    \centering
    \begin{subfigure}{0.49\textwidth}
        \includegraphics[width=\textwidth]{../pictures/originalDirac.eps}
        \caption{Actual Diracs.}
        \label{fig:ex7_7_0}
    \end{subfigure}
    ~
    \begin{subfigure}{0.49\textwidth}
        \includegraphics[width=\textwidth]{../pictures/ex7_7_1.eps}
        \caption{$\sigma^2 = 0.001$.}
        \label{fig:ex7_7_1}
    \end{subfigure}
    \\
    \begin{subfigure}{0.49\textwidth}
        \includegraphics[width=\textwidth]{../pictures/ex7_7_2.eps}
        \caption{$\sigma^2 = 0.01$.}
        \label{fig:ex7_7_2}
    \end{subfigure}
    ~
    \begin{subfigure}{0.49\textwidth}
        \includegraphics[width=\textwidth]{../pictures/ex7_7_3.eps}
        \caption{$\sigma^2 = 0.1$.}
        \label{fig:ex7_7_3}
    \end{subfigure}
    \\
    \begin{subfigure}{0.49\textwidth}
        \includegraphics[width=\textwidth]{../pictures/ex7_7_4.eps}
        \caption{$\sigma^2 = 1$.}
        \label{fig:ex7_7_4}
    \end{subfigure}
    ~
    \begin{subfigure}{0.49\textwidth}
        \includegraphics[width=\textwidth]{../pictures/ex7_7_5.eps}
        \caption{$\sigma^2 = 10$.}
        \label{fig:ex7_7_5}
    \end{subfigure}

    \caption{Estimate of Diracs with sampling when $N = 7$, with normal annihilation filter.}
    \label{fig:ex7_7}
\end{figure}

\begin{figure}[H]
    \captionsetup[subfigure]{position=b}
    \centering
    \begin{subfigure}{0.49\textwidth}
        \includegraphics[width=\textwidth]{../pictures/originalDirac.eps}
        \caption{Actual Diracs.}
        \label{fig:ex7_8_0}
    \end{subfigure}
    ~
    \begin{subfigure}{0.49\textwidth}
        \includegraphics[width=\textwidth]{../pictures/ex7_8_1.eps}
        \caption{$\sigma^2 = 0.001$.}
        \label{fig:ex7_8_1}
    \end{subfigure}
    \\
    \begin{subfigure}{0.49\textwidth}
        \includegraphics[width=\textwidth]{../pictures/ex7_8_2.eps}
        \caption{$\sigma^2 = 0.01$.}
        \label{fig:ex7_8_2}
    \end{subfigure}
    ~
    \begin{subfigure}{0.49\textwidth}
        \includegraphics[width=\textwidth]{../pictures/ex7_8_3.eps}
        \caption{$\sigma^2 = 0.1$.}
        \label{fig:ex7_8_3}
    \end{subfigure}
    \\
    \begin{subfigure}{0.49\textwidth}
        \includegraphics[width=\textwidth]{../pictures/ex7_8_4.eps}
        \caption{$\sigma^2 = 1$.}
        \label{fig:ex7_8_4}
    \end{subfigure}
    ~
    \begin{subfigure}{0.49\textwidth}
        \includegraphics[width=\textwidth]{../pictures/ex7_8_5.eps}
        \caption{$\sigma^2 = 10$.}
        \label{fig:ex7_8_5}
    \end{subfigure}

    \caption{Estimate of Diracs with sampling when $N = 8$, with normal annihilation filter.}
    \label{fig:ex7_8}
\end{figure}

%%%%%%%%% TLS

\begin{figure}[H]
    \captionsetup[subfigure]{position=b}
    \centering
    \begin{subfigure}{0.49\textwidth}
        \includegraphics[width=\textwidth]{../pictures/originalDirac.eps}
        \caption{Actual Diracs.}
        \label{fig:ex7_tls_5_0}
    \end{subfigure}
    ~
    \begin{subfigure}{0.49\textwidth}
        \includegraphics[width=\textwidth]{../pictures/ex7_tls_5_1.eps}
        \caption{$\sigma^2 = 0.001$.}
        \label{fig:ex7_tls_5_1}
    \end{subfigure}
    \\
    \begin{subfigure}{0.49\textwidth}
        \includegraphics[width=\textwidth]{../pictures/ex7_tls_5_2.eps}
        \caption{$\sigma^2 = 0.01$.}
        \label{fig:ex7_tls_5_2}
    \end{subfigure}
    ~
    \begin{subfigure}{0.49\textwidth}
        \includegraphics[width=\textwidth]{../pictures/ex7_tls_5_3.eps}
        \caption{$\sigma^2 = 0.1$.}
        \label{fig:ex7_tls_5_3}
    \end{subfigure}
    \\
    \begin{subfigure}{0.49\textwidth}
        \includegraphics[width=\textwidth]{../pictures/ex7_tls_5_4.eps}
        \caption{$\sigma^2 = 1$.}
        \label{fig:ex7_tls_5_4}
    \end{subfigure}
    ~
    \begin{subfigure}{0.49\textwidth}
        \includegraphics[width=\textwidth]{../pictures/ex7_tls_5_5.eps}
        \caption{$\sigma^2 = 10$.}
        \label{fig:ex7_tls_5_5}
    \end{subfigure}

    \caption{Estimate of Diracs with sampling when $N = 5$, with annihilation filter and TLS.}
    \label{fig:ex7_tls_5}
\end{figure}

\begin{figure}[H]
    \captionsetup[subfigure]{position=b}
    \centering
    \begin{subfigure}{0.49\textwidth}
        \includegraphics[width=\textwidth]{../pictures/originalDirac.eps}
        \caption{Actual Diracs.}
        \label{fig:ex7_tls_6_0}
    \end{subfigure}
    ~
    \begin{subfigure}{0.49\textwidth}
        \includegraphics[width=\textwidth]{../pictures/ex7_tls_6_1.eps}
        \caption{$\sigma^2 = 0.001$.}
        \label{fig:ex7_tls_6_1}
    \end{subfigure}
    \\
    \begin{subfigure}{0.49\textwidth}
        \includegraphics[width=\textwidth]{../pictures/ex7_tls_6_2.eps}
        \caption{$\sigma^2 = 0.01$.}
        \label{fig:ex7_tls_6_2}
    \end{subfigure}
    ~
    \begin{subfigure}{0.49\textwidth}
        \includegraphics[width=\textwidth]{../pictures/ex7_tls_6_3.eps}
        \caption{$\sigma^2 = 0.1$.}
        \label{fig:ex7_tls_6_3}
    \end{subfigure}
    \\
    \begin{subfigure}{0.49\textwidth}
        \includegraphics[width=\textwidth]{../pictures/ex7_tls_6_4.eps}
        \caption{$\sigma^2 = 1$.}
        \label{fig:ex7_tls_6_4}
    \end{subfigure}
    ~
    \begin{subfigure}{0.49\textwidth}
        \includegraphics[width=\textwidth]{../pictures/ex7_tls_6_5.eps}
        \caption{$\sigma^2 = 10$.}
        \label{fig:ex7_tls_6_5}
    \end{subfigure}

    \caption{Estimate of Diracs with sampling when $N = 6$, with annihilation filter and TLS.}
    \label{fig:ex7_tls_6}
\end{figure}

\begin{figure}[H]
    \captionsetup[subfigure]{position=b}
    \centering
    \begin{subfigure}{0.49\textwidth}
        \includegraphics[width=\textwidth]{../pictures/originalDirac.eps}
        \caption{Actual Diracs.}
        \label{fig:ex7_tls_7_0}
    \end{subfigure}
    ~
    \begin{subfigure}{0.49\textwidth}
        \includegraphics[width=\textwidth]{../pictures/ex7_tls_7_1.eps}
        \caption{$\sigma^2 = 0.001$.}
        \label{fig:ex7_tls_7_1}
    \end{subfigure}
    \\
    \begin{subfigure}{0.49\textwidth}
        \includegraphics[width=\textwidth]{../pictures/ex7_tls_7_2.eps}
        \caption{$\sigma^2 = 0.01$.}
        \label{fig:ex7_tls_7_2}
    \end{subfigure}
    ~
    \begin{subfigure}{0.49\textwidth}
        \includegraphics[width=\textwidth]{../pictures/ex7_tls_7_3.eps}
        \caption{$\sigma^2 = 0.1$.}
        \label{fig:ex7_tls_7_3}
    \end{subfigure}
    \\
    \begin{subfigure}{0.49\textwidth}
        \includegraphics[width=\textwidth]{../pictures/ex7_tls_7_4.eps}
        \caption{$\sigma^2 = 1$.}
        \label{fig:ex7_tls_7_4}
    \end{subfigure}
    ~
    \begin{subfigure}{0.49\textwidth}
        \includegraphics[width=\textwidth]{../pictures/ex7_tls_7_5.eps}
        \caption{$\sigma^2 = 10$.}
        \label{fig:ex7_tls_7_5}
    \end{subfigure}

    \caption{Estimate of Diracs with sampling when $N = 7$, with annihilation filter and TLS.}
    \label{fig:ex7_tls_7}
\end{figure}

\begin{figure}[H]
    \captionsetup[subfigure]{position=b}
    \centering
    \begin{subfigure}{0.49\textwidth}
        \includegraphics[width=\textwidth]{../pictures/originalDirac.eps}
        \caption{Actual Diracs.}
        \label{fig:ex7_tls_8_0}
    \end{subfigure}
    ~
    \begin{subfigure}{0.49\textwidth}
        \includegraphics[width=\textwidth]{../pictures/ex7_tls_8_1.eps}
        \caption{$\sigma^2 = 0.001$.}
        \label{fig:ex7_tls_8_1}
    \end{subfigure}
    \\
    \begin{subfigure}{0.49\textwidth}
        \includegraphics[width=\textwidth]{../pictures/ex7_tls_8_2.eps}
        \caption{$\sigma^2 = 0.01$.}
        \label{fig:ex7_tls_8_2}
    \end{subfigure}
    ~
    \begin{subfigure}{0.49\textwidth}
        \includegraphics[width=\textwidth]{../pictures/ex7_tls_8_3.eps}
        \caption{$\sigma^2 = 0.1$.}
        \label{fig:ex7_tls_8_3}
    \end{subfigure}
    \\
    \begin{subfigure}{0.49\textwidth}
        \includegraphics[width=\textwidth]{../pictures/ex7_tls_8_4.eps}
        \caption{$\sigma^2 = 1$.}
        \label{fig:ex7_tls_8_4}
    \end{subfigure}
    ~
    \begin{subfigure}{0.49\textwidth}
        \includegraphics[width=\textwidth]{../pictures/ex7_tls_8_5.eps}
        \caption{$\sigma^2 = 10$.}
        \label{fig:ex7_tls_8_5}
    \end{subfigure}

    \caption{Estimate of Diracs with sampling when $N = 8$, with annihilation filter and TLS.}
    \label{fig:ex7_tls_8}
\end{figure}

%%%%%%%%% TLS CAD

\begin{figure}[H]
    \captionsetup[subfigure]{position=b}
    \centering
    \begin{subfigure}{0.49\textwidth}
        \includegraphics[width=\textwidth]{../pictures/originalDirac.eps}
        \caption{Actual Diracs.}
        \label{fig:ex7_tlscad_5_0}
    \end{subfigure}
    ~
    \begin{subfigure}{0.49\textwidth}
        \includegraphics[width=\textwidth]{../pictures/ex7_tlscad_5_1.eps}
        \caption{$\sigma^2 = 0.001$.}
        \label{fig:ex7_tlscad_5_1}
    \end{subfigure}
    \\
    \begin{subfigure}{0.49\textwidth}
        \includegraphics[width=\textwidth]{../pictures/ex7_tlscad_5_2.eps}
        \caption{$\sigma^2 = 0.01$.}
        \label{fig:ex7_tlscad_5_2}
    \end{subfigure}
    ~
    \begin{subfigure}{0.49\textwidth}
        \includegraphics[width=\textwidth]{../pictures/ex7_tlscad_5_3.eps}
        \caption{$\sigma^2 = 0.1$.}
        \label{fig:ex7_tlscad_5_3}
    \end{subfigure}
    \\
    \begin{subfigure}{0.49\textwidth}
        \includegraphics[width=\textwidth]{../pictures/ex7_tlscad_5_4.eps}
        \caption{$\sigma^2 = 1$.}
        \label{fig:ex7_tlscad_5_4}
    \end{subfigure}
    ~
    \begin{subfigure}{0.49\textwidth}
        \includegraphics[width=\textwidth]{../pictures/ex7_tlscad_5_5.eps}
        \caption{$\sigma^2 = 10$.}
        \label{fig:ex7_tlscad_5_5}
    \end{subfigure}

    \caption{Estimate of Diracs with sampling when $N = 5$, with annihilation filter, TLS, and Cadzow.}
    \label{fig:ex7_tlscad_5}
\end{figure}

\begin{figure}[H]
    \captionsetup[subfigure]{position=b}
    \centering
    \begin{subfigure}{0.49\textwidth}
        \includegraphics[width=\textwidth]{../pictures/originalDirac.eps}
        \caption{Actual Diracs.}
        \label{fig:ex7_tlscad_6_0}
    \end{subfigure}
    ~
    \begin{subfigure}{0.49\textwidth}
        \includegraphics[width=\textwidth]{../pictures/ex7_tlscad_6_1.eps}
        \caption{$\sigma^2 = 0.001$.}
        \label{fig:ex7_tlscad_6_1}
    \end{subfigure}
    \\
    \begin{subfigure}{0.49\textwidth}
        \includegraphics[width=\textwidth]{../pictures/ex7_tlscad_6_2.eps}
        \caption{$\sigma^2 = 0.01$.}
        \label{fig:ex7_tlscad_6_2}
    \end{subfigure}
    ~
    \begin{subfigure}{0.49\textwidth}
        \includegraphics[width=\textwidth]{../pictures/ex7_tlscad_6_3.eps}
        \caption{$\sigma^2 = 0.1$.}
        \label{fig:ex7_tlscad_6_3}
    \end{subfigure}
    \\
    \begin{subfigure}{0.49\textwidth}
        \includegraphics[width=\textwidth]{../pictures/ex7_tlscad_6_4.eps}
        \caption{$\sigma^2 = 1$.}
        \label{fig:ex7_tlscad_6_4}
    \end{subfigure}
    ~
    \begin{subfigure}{0.49\textwidth}
        \includegraphics[width=\textwidth]{../pictures/ex7_tlscad_6_5.eps}
        \caption{$\sigma^2 = 10$.}
        \label{fig:ex7_tlscad_6_5}
    \end{subfigure}

    \caption{Estimate of Diracs with sampling when $N = 6$, with annihilation filter, TLS, and Cadzow.}
    \label{fig:ex7_tlscad_6}
\end{figure}

\begin{figure}[H]
    \captionsetup[subfigure]{position=b}
    \centering
    \begin{subfigure}{0.49\textwidth}
        \includegraphics[width=\textwidth]{../pictures/originalDirac.eps}
        \caption{Actual Diracs.}
        \label{fig:ex7_tlscad_7_0}
    \end{subfigure}
    ~
    \begin{subfigure}{0.49\textwidth}
        \includegraphics[width=\textwidth]{../pictures/ex7_tlscad_7_1.eps}
        \caption{$\sigma^2 = 0.001$.}
        \label{fig:ex7_tlscad_7_1}
    \end{subfigure}
    \\
    \begin{subfigure}{0.49\textwidth}
        \includegraphics[width=\textwidth]{../pictures/ex7_tlscad_7_2.eps}
        \caption{$\sigma^2 = 0.01$.}
        \label{fig:ex7_tlscad_7_2}
    \end{subfigure}
    ~
    \begin{subfigure}{0.49\textwidth}
        \includegraphics[width=\textwidth]{../pictures/ex7_tlscad_7_3.eps}
        \caption{$\sigma^2 = 0.1$.}
        \label{fig:ex7_tlscad_7_3}
    \end{subfigure}
    \\
    \begin{subfigure}{0.49\textwidth}
        \includegraphics[width=\textwidth]{../pictures/ex7_tlscad_7_4.eps}
        \caption{$\sigma^2 = 1$.}
        \label{fig:ex7_tlscad_7_4}
    \end{subfigure}
    ~
    \begin{subfigure}{0.49\textwidth}
        \includegraphics[width=\textwidth]{../pictures/ex7_tlscad_7_5.eps}
        \caption{$\sigma^2 = 10$.}
        \label{fig:ex7_tlscad_7_5}
    \end{subfigure}

    \caption{Estimate of Diracs with sampling when $N = 7$, with annihilation filter, TLS, and Cadzow.}
    \label{fig:ex7_tlscad_7}
\end{figure}

\begin{figure}[H]
    \captionsetup[subfigure]{position=b}
    \centering
    \begin{subfigure}{0.49\textwidth}
        \includegraphics[width=\textwidth]{../pictures/originalDirac.eps}
        \caption{Actual Diracs.}
        \label{fig:ex7_tlscad_8_0}
    \end{subfigure}
    ~
    \begin{subfigure}{0.49\textwidth}
        \includegraphics[width=\textwidth]{../pictures/ex7_tlscad_8_1.eps}
        \caption{$\sigma^2 = 0.001$.}
        \label{fig:ex7_tlscad_8_1}
    \end{subfigure}
    \\
    \begin{subfigure}{0.49\textwidth}
        \includegraphics[width=\textwidth]{../pictures/ex7_tlscad_8_2.eps}
        \caption{$\sigma^2 = 0.01$.}
        \label{fig:ex7_tlscad_8_2}
    \end{subfigure}
    ~
    \begin{subfigure}{0.49\textwidth}
        \includegraphics[width=\textwidth]{../pictures/ex7_tlscad_8_3.eps}
        \caption{$\sigma^2 = 0.1$.}
        \label{fig:ex7_tlscad_8_3}
    \end{subfigure}
    \\
    \begin{subfigure}{0.49\textwidth}
        \includegraphics[width=\textwidth]{../pictures/ex7_tlscad_8_4.eps}
        \caption{$\sigma^2 = 1$.}
        \label{fig:ex7_tlscad_8_4}
    \end{subfigure}
    ~
    \begin{subfigure}{0.49\textwidth}
        \includegraphics[width=\textwidth]{../pictures/ex7_tlscad_8_5.eps}
        \caption{$\sigma^2 = 10$.}
        \label{fig:ex7_tlscad_8_5}
    \end{subfigure}

    \caption{Estimate of Diracs with sampling when $N = 8$, with annihilation filter, TLS, and Cadzow.}
    \label{fig:ex7_tlscad_8}
\end{figure}

\newpage

\section{Exercise 8 Code}
Note:ImageFusion was modified, but only to deal with the \mcode{edgetaper} bug. Also, \mcode{Main_SR} was too, but only to pass on the limit variable to the \mcode{ImageRegistration}. \mcode{ex8.m} contains some testing scripts for the optimum SNR.

\subsection{ImageRegistration.m}
\lstinputlisting{../ImageRegistration.m}
\newpage

\subsection{Actual Script}
\lstinputlisting{../ex8.m}

\end{document}
